\documentclass[a4paper,11pt,twoside]{report}
\usepackage[T1]{fontenc}
\usepackage[frenchb]{babel} %active le mode français
\usepackage[latin1,utf8]{inputenc} % Mettre des accents
\usepackage[top=2cm , bottom=2cm , left=2cm , right=2cm]{geometry} %propriétés de notre page
\usepackage{amsmath} %liste de symboles et applications mathématiques
\usepackage{amsfonts} %idem
\usepackage{color} %Permet d'utiliser la couleur dans nos documents
\usepackage[usenames,dvipsnames]{xcolor}
\usepackage{listings} %Paquet de coloration syntaxique (langages)
\usepackage{hyperref} % Créer des liens et des signets 
\usepackage[babel=true]{csquotes} %permet les quotations (guillemets)
\usepackage{graphicx} %Importation d'image
\usepackage{fancyhdr} %en-tête et pieds de page+
\usepackage{lastpage} %permet d'obtenir le nombre total de page
\usepackage{multido}
\usepackage{eurosym}
\usepackage{listingsutf8}
\usepackage{tikz}
\usepackage{ulem}
\usepackage{colortbl} %permet de mettre de personnaliser les colones des tabular
\usepackage{pdfpages}
\usepackage{appendix} %annexes
\setlength{\headheight}{15pt}
\lstset{language=C,
	breaklines=true,
	numbers=left,
	keywordstyle=\color{blue},
	commentstyle=\color{Gray},
	stringstyle=\color{red},
	tabsize=4,
	framexleftmargin=7mm,
	frame=single
	}
% Informations du rapport
\title {Rapport de TP \\ Problème de production de type ULS}
\author {Quentin Tonneau - Camille Drancourt}
\date{}
\renewcommand{\thesection}{\arabic{section}}
%Propriétés des liens
\hypersetup{
colorlinks=true, %colorise les liens  
urlcolor= blue, %couleur des hyperliens 
linkcolor= blue,%couleur des liens internes
citecolor=black
} 
%\setlength{\parindent}{0cm}
\addto\captionsfrench{
	\renewcommand{\chaptername}{Partie}
}

\fancypagestyle{plain}
{

	\fancyhead{}
	\fancyfoot{}
	\lhead{Quentin Tonneau - Camille Drancourt}
	\rhead{Université de Nantes}
	\chead{Problème de production de type ULS}
}
\begin{document}
\tableofcontents
\newpage

\chapter{Description du problème}
Le problème de planification de production étudié dans cet exercice est appelé \textit{Uncapacitated Lot-Sizing Problem}. Comme son nom l'indique, il n'y a dans ce cas précis aucune contrainte de capacité ou de limite de production.
Le plan de production est décomposé en T instances, dont on cherche à décider l'ouverture ou non ($y_t$), ainsi que la quantité d'éléments à produire $x_t$. L'ouverture d'une instance entraîne un coût fixe ($f_t$), ainsi qu'un coût unitaire de production ($p_t$).
Il faut impérativement subvenir aux demandes des clients pour tout instant $t$, mais il est possible à une instance $t$ de produire plus que nécessaire, et stocker l'ensemble de la ``surproduction'' ($s_t$) jusqu'à $t+1$ moyennant un coût unitaire de $h_t$.


Nous tentons dans cette analyse de résoudre quelques instances de ce problème via un algorithme de \textbf{branch \& bound}, et nous essayons de comprendre quelles sont les astuces de modélisation et les choix de résolution permettant d'améliorer au fur et à mesure notre rapidité de résolution.

\chapter{Le projet initial}
\section{La modélisation}
Les algorithmes de branch \& bound sont très utiles pour résoudre des problèmes en variables entières, en effectuant une série de résolution du problème ``relaché'' et en introduisant à chaque étape une nouvelle contrainte d'intégrité sur une des variables de décision.
Nous introduisons donc un modèle correspondant au problème ci-dessus, mais dont les variables entières ou binaires ($y_t$) ne possèdent pas de contraintes d’intégrité.

Voici le modèle proposé :
\[
 \left[ \begin{matrix}
         min z= & \displaystyle \sum_{t\in \{1...T\}} y_t \times f_t + x_t \times p_t + s_t \times h_t \\
         s/c & s_{t-1}+x_{t}& =d_t+s_t & \forall t\in \{0...T\} \\
         & y_t*M &\geq x_t \\
         &s_0&=0\\
         &s_t&=0\\
         &y_t,x_t,s_t&\geq 0
        \end{matrix}
 \right]
\]
La première contrainte, que nous appelons contrainte d'équilibre, impose l'absence de perte de produit lors de la production. En effet, l'ensemble des éléments entrants (stock précédent et production) soit être égal à l'ensemble des éléments sortants (ventes et stock).
La seconde contrainte implémente une constante M très grande imposant l'ouverture de l'instance si au moins un élément est produit par cette instance.
Les contraintes (3) et (4) indiquent l'état du stock au départ et à l'arrivé, fixés ici à 0.

\section{L'algorithme}
Pour résoudre entièrement le problème, nous avons exécuté à la main l'algorithme du branch \& bound en effectuant les choix suivants :
\begin{itemize}
 \item Relâcher ou contraindre les variables $y_t$ à 0 ou 1.
 \item Toujours choisir la variable $y_i$ la plus proche de 0 et la fixer à 0 (dans un premier temps)
\end{itemize}

\section{L'implémentation}
Nous utilisons pour la résolution le solveur GLPK sous sa forme ``lpsolve''. L'implémentation de la modélisation dans ce langage est donné en Annexe \ref{implementationlpsolve}
Afin de contraindre une variable, nous ajoutons à chaque étape une ligne de type :
\begin{verbatim}
 s.t. contrX :y[A]=B;
\end{verbatim}
 avec A=$t$ et B$\in \{0,1\}$

\section{Résultats}
%TODO : Dessiner l'arbre obtenu



%TODO : Sections : amélioration de M (valeur fixe, puis tableau de valeurs)

%TODO : Sections : contraintes additionnelles (coupes)

%TODO : section  ; analyse et dessin avec branches directes

%TODO : Explications Wagner-Whistin

%TODO : Problème final

%TODO : conclusion









\begin{appendices}
\chapter{Implémentation lpsolve}
\label{implementationlpsolve}
\begin{verbatim}
 #Données
param nbpostes;

set T:=1..nbpostes;
set T2:=0..nbpostes;
param M:=25;

param d{T};
param h{T};
param p{T};
param f{T};

#Variables de décision
var x{T} >=0;
var y{T} >=0;
var s{T2}>=0;

#Fonction objectif
minimize couts : sum{i in T} (y[i]*f[i] + x[i]*p[i] + h[i]*s[i]);

#Contraintes

s.t. Equilibre{i in T} : (s[i-1]+x[i])=(s[i]+d[i]);
s.t. Activation{i in T}: y[i]*sum{j in i..nbpostes}d[j] >= x[i];
s.t. contr1 :s[0]=0;
s.t. contr2 :s[nbpostes]=0;

#Résolution
solve;

#Affichange des res
display : s;
display : x;
display : y;
display : sum{i in T} (y[i]*f[i] + x[i]*p[i] + h[i]*s[i]); #fonction objectif à recopier

data;

param nbpostes:= 5;
param d:= 1 3 2 5 3 6 4 3 5 8;
param p:= 1 2 2 4 3 6 4 8 5 10 ;
param h:= 1 3 2 2 3 3 4 2 5 0;
param f:= 1 10 2 8 3 6 4 4 5 2;

end;
\end{verbatim}
\end{appendices}

\end{document}

